%========================================
% 英文摘要

\phantomsection % 让加入的目录带超链接
\addcontentsline{toc}{section}{Abstract(Key words)} % 在目录中显示摘要

\renewcommand\abstractname{{\fontsize{18}{21.6} \selectfont Design and Implementation of IoT-Based Distributed Data Acquisition System}}

\begin{abstract}
%\vspace{1em}
\end{abstract}
%\begin{spacing}{1.3}
{\noindent \fontsize{12}{14.4} \selectfont \textbf{【Abstract】}}
{\fontsize{10.5}{12.6} \selectfont
With the rapid development of technology of IoT(Internet of Things),the data produced and those need to be stored must increase eruptively. Consequently, to handle the task of acquiring and storage the big capacity of data, a high scaling,high available and high concurrent distributed data acquisition system is needed,in this case,people are able to mine these data and provide people with smarter service. Given that the type of devices and data format are keep growing, it is a good solution to use RESTful Architecture as the guidance of designing and implementing of the system. Furthermore,considering the need of three elements listed before for acquiring data,this article will focus on combining Nginx,Nodejs,MongoDB to implement the distributed data acquisition system and realize performance evaluation.
}

\vspace{1em}

{\noindent \fontsize{12}{14.4} \selectfont \textbf{【Key words】}}
{\fontsize{10.5}{12.6} \selectfont
Restful;Nodejs;IoT;Distributed System;Nginx;MongoDB\newline
\newline
}


%\end{spacing}
~~~~~~~~~~~~~~~~~~~~~~~~~~~~~~~~~~~~~~~~~~~~~~~~~~~~~~~~~~~~~~~~~~~~~~~~~~~~~~~~~~~~~~~~~~~~~~~~~~~~~~~~~~~~~~~~~~ 指导老师:陈亮讲师

