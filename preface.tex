\iffalse
\bibliography{myref}
\fi
\section{绪论}

\subsection{研究背景和意义}


\subsection{国内外发展现状}

\subsection{主要研究内容}

\section{相关技术发展}

\subsection{Python}

Python是一种面向对象、直译式的计算机程序语言,具有近二十年的发展历史。它包含了一组功能完备的标准库,能够轻松完成很多常见的任务。它的语法简单,与其它大多数程序设计语言使用大括号不一样,它使用缩进来定义语句块。

与Scheme、Ruby、Perl、Tcl等动态语言一样,Python具备垃圾回收功能,能够自动管理内存使用。它经常被当作脚本语言用于处理系统管理任务和网络程序编写,然而它也非常适合完成各种高级任务。Python虚拟机本身几乎可以在所有的作业系统中运行。使用一些诸如py2exe、PyPy、PyInstaller之类的工具可以将Python源代码转换成可以脱离Python解释器运行的程序。

Python的官方解释器是CPython,该解释器用C语言编写,是一个由社区驱动的自由软件,目前由Python软件基金会管理。

Python支持命令式程序设计、面向对象程序设计、函数式编程、面向侧面的程序设计、泛型编程多种编程范式。\cite{python}

\subsection{MongoDB}

MongoDB是一个跨平台的文件导向型系统。MongoDB归属于NoSQL数据库,它没有使用传统的基于表的关系型数据库的结构,而是用了一种JSON格式的文档,还有动态的schemas(MongoDB称之为BSON),这让相似类型的数据的整合变得十分简单和快速。MongoDB还是完全免费,开源的软件。

MongoDB最早是由10gen公司(现在叫MongoDB公司)在2007年10月开发出来的,当时只是一个叫platform as service产品的部件。2009年被开源出来,然后由10gen提供商业支持和其它服务。从那以后,MongoDB开始成为许多网站和服务的后台数据库,包括Craigslist, eBay, Foursquare, SourceForge, Viacom, 和the New York Times等。现在,MongoDB已经是最流行的NoSQL数据库了。\cite{mongodb}

\subsection{Web app}

Web app指的是跑在浏览器端的软件,一般使用JavaScript,HTML和CSS开发的,并依靠浏览器运行。浏览器的普遍性应该是Web app流行的最大原因。因为是借助于浏览器,所以Web app的安装和更新都十分简单,不需要很繁琐的步骤。这也很大程度推动了Web app的发展。并且Web app还是跨平台的,你没必要因为换了一个平台而要重新选择对应版本的软件。现在普遍的Web app包括邮件客户端,在线零售店,在线拍卖,wiki系统等。

\subsubsection{HTML5}

HTML5是HTML最新的修订版本,2014年10月由万维网联盟(W3C)完成标准制定。目标是取代1999年所制定的HTML 4.01和XHTML 1.0标准,以期能在互联网应用迅速发展的时候,使网络标准达到符合当代的网络需求。广义论及HTML5时,实际指的是包括HTML、CSS和JavaScript在内的一套技术组合。它希望能够减少网页浏览器对于需要插件的丰富性网络应用服务(Plug-in-Based Rich Internet Application,RIA),例如:Adobe Flash、Microsoft Silverlight与Oracle JavaFX的需求,并且提供更多能有效加强网络应用的标准集。

具体来说,HTML5添加了许多新的语法特征,其中包括<video>、<audio>和<canvas>元素,同时集成了SVG内容。这些元素是为了更容易的在网页中添加和处理多媒体和图片内容而添加的。其它新的元素如<section>、<article>、<header>和<nav>则是为了丰富文档的数据内容。新的属性的添加也是为了同样的目的。同时也有一些属性和元素被移除掉了。一些元素,像<a>、<cite>和<menu>被修改,重新定义或标准化了。同时APIs和DOM已经成为HTML5中的基础部分了。HTML5还定义了处理非法文档的具体细节,使得所有浏览器和客户端程序能够一致地处理语法错误。

\subsubsection{JavaScript}

JavaScript,一种直译式脚本语言,是一种动态类型、弱类型、基于原型的语言,内置支持类。它的解释器被称为JavaScript引擎,为浏览器的一部分,广泛用于客户端的脚本语言,最早是在HTML网页上使用,用来给HTML网页增加动态功能。然而现在JavaScript也可被用于网络服务器,如Node.js。

在1995年时,由网景公司的布兰登·艾克,在网景导航者浏览器上首次设计实现而成。因为网景公司与昇阳公司合作,网景公司管理层次结构希望它外观看起来像Java,因此取名为JavaScript。但实际上它的语义与Self及Scheme较为接近。

为了获取技术优势,微软推出了JScript,与JavaScript同样可在浏览器上运行。为了统一规格,1997年,在ECMA(欧洲计算机制造商协会)的协调下,由网景、昇阳、微软和Borland公司组成的工作组确定统一标准:ECMA-262。因为JavaScript兼容于ECMA标准,因此也称为ECMAScript。
