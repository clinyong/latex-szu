%========================================
%		Packages used in this template
\usepackage{xeCJK}			% 中文支持
\usepackage{titlesec}  %用于修改英文章节为中文章节等
%\titleformat{\section}[block]{\large\bf}{\thesection}{10pt}{}  
\usepackage{fontspec}
\usepackage[justification=centering]{caption} %让figure的标题居中
\usepackage{underscore}     % 处理下划线
\usepackage{pdfpages}		% 插入pdf
\usepackage{graphicx}		% 图形支持
\usepackage{amssymb, amsmath}	% 数学符号
\usepackage{fancyhdr}		% 页眉设置
\usepackage{lastpage}       % 引用最后一页,即算出总页数
\usepackage{cite}			% 引用文献
%%%%%%%%%%%%%%%%%%  上角标引用并加括号命令  %%%%%%%%%%%%%%%%%%%%%%%%%%%%%%%
\makeatletter
\def\@cite#1#2{\textsuperscript{[{#1\if@tempswa , #2\fi}]}}
\makeatother
%%%%%%%%%%%%%%%%%%  上角标引用并加括号命令  %%%%%%%%%%%%%%%%%%%%%%%%%%%%%%%


\usepackage{indentfirst}	% 首行缩进
\usepackage[]{hyperref}		% 让tableofcontents支持超链接
%\usepackage[top=1in,bottom=1in,left=1.4in,right=1.2in]{geometry}	% 设置页边距
\usepackage{booktabs}       % 增加表格效果
\usepackage{listings} %插入代码块
% listings 源代码显示宏包
\usepackage{listings}

\lstset{
%language={[ISO]C++},       %language为,还有{[Visual]C++}  
%alsolanguage=[ANSI]C,      %可以添加很多个alsolanguage,如alsolanguage=matlab,alsolanguage=VHDL等  
%alsolanguage= tcl,  
%alsolanguage= XML,  
tabsize=4, %  
  %frame=single,%普通边框   shadowbox, %把代码用带有阴影的框圈起来  
  commentstyle=\color{red!50!green!50!blue!50},%浅灰色的注释  
  %rulesepcolor=\color{red!20!green!20!blue!20},%代码块边框为淡青色  
  keywordstyle=\color{blue!90}\bfseries, %代码关键字的颜色为蓝色,粗体  
  showstringspaces=false,%不显示代码字符串中间的空格标记  
  stringstyle=\ttfamily, % 代码字符串的特殊格式  
  keepspaces=true, %  
  breakindent=22pt, % 
  numbers=none,%left,左侧显示行号 往左靠,还可以为right,或none,即不加行号  
  %stepnumber=1,%若设置为2,则显示行号为1,3,5,即stepnumber为公差,默认stepnumber=1  
  %numberstyle=\tiny, %行号字体用小号  
  %numberstyle={\scriptsize} ,%设置行号的大小,大小有tiny,scriptsize,footnotesize,small,normalsize,large等  
  %numbersep=8pt,  %设置行号与代码的距离,默认是5pt  
  basicstyle=\ttfamily, %\footnotesize. % 这句设置代码的大小  
  showspaces=false, %  
  %flexiblecolumns=true, %  
  breaklines=true, %对过长的代码自动换行  
  breakautoindent=true,%  
  breakindent=4em, %  
  %saboveskip=1em, %代码块边框  
  tabsize=2,  
  showstringspaces=false, %不显示字符串中的空格  
  %backgroundcolor=\color[RGB]{245,245,244},   %代码背景色  
  %backgroundcolor=\color[rgb]{0.91,0.91,0.91}    %添加背景色  
  %escapeinside=``,  %在``里显示中文  
  %% added by http://bbs.ctex.org/viewthread.php?tid=53451  
  fontadjust,  
  %captionpos=t,  
  framextopmargin=2pt,framexbottommargin=2pt,abovecaptionskip=-3pt,belowcaptionskip=3pt,  
  xleftmargin=0em,xrightmargin=0em, % 设定listing左右的空白  xleftmargin=2em xrightmargin=2em
  %texcl=true,  
  % 设定中文冲突,断行,列模式,数学环境输入,listing数字的样式  
  %extendedchars=false,columns=flexible,mathescape=true  
  % numbersep=-1em  
  %% tabsize=4, %
  %frame=shadowbox, %把代码用带有阴影的框圈起来
  %commentstyle=\color{red!50!green!50!blue!50},%浅灰色的注释
  %%rulesepcolor=\color{red!20!green!20!blue!20},%代码块边框为淡青色
  %%keywordstyle=\color{blue!90}\bfseries, %代码关键字的颜色为蓝色,粗体
  %showstringspaces=false,%不显示代码字符串中间的空格标记
  %%stringstyle=\ttfamily, % 代码字符串的特殊格式
  %%keepspaces=true, %
  %%breakindent=22pt, %
  %numbers=left,%左侧显示行号
  %stepnumber=1,%
  %numberstyle=\tiny, %行号字体用小号
  %%basicstyle=\footnotesize, %
  %showspaces=false, %
  %%flexiblecolumns=true, %
  %breaklines=true, %对过长的代码自动换行
  %%breakautoindent=true,%
  %%breakindent=4em, %
  %%escapebegin=\begin{CJK*}{GBK}{hei},escapeend=\end{CJK*},
  %aboveskip=1em, %代码块边框
  %%%% added by http://bbs.ctex.org/viewthread.php?tid=53451
  %fontadjust,
  %%captionpos=t,
  %%framextopmargin=2pt,framexbottommargin=2pt,abovecaptionskip=-3pt,belowcaptionskip=3pt,
  %xleftmargin=4em,xrightmargin=4em % 设定listing左右的空白
  %%texcl=true,
  %%% 设定中文冲突,断行,列模式,数学环境输入,listing数字的样式
  %%extendedchars=false,columns=flexible,mathescape=true
  %%% numbersep=-1em
}
\usepackage{setspace} % 调整行距
\usepackage{url}      % 参考文献引用网页
%\usepackage[top=0.8in,bottom=0.8in,left=1.2in,right=0.6in]{geometry}  %设置页边距(学校的要求)
\usepackage[top=1in,bottom=1in,left=1.4in,right=1.2in]{geometry}
%========================================
%		Settings
\setmainfont{Times New Roman} % 英文字体样式
\setCJKmainfont[BoldFont={SimHei}]{SimSun}      % 中文字体设置为宋体
\setCJKfamilyfont{ZenHei}{SimHei} % Use \CJKfamily{ZenHei} where you need.
\setCJKfamilyfont{ZS}{STZhongsong} % 华文中宋
\setCJKfamilyfont{KT}{KaiTi} % 楷体

\setlength{\parindent}{2em}	% 首行缩进,2字符
\numberwithin{equation}{section}   % 使公式标号为 3.1 的形式

\setlength{\parskip}{0.6\baselineskip} % 设置段间距离
\renewcommand{\baselinestretch}{1.3} % 设置行间距离

%========================================
%		Redefine commands
\newcommand{\chapter}{第\CJKnumber{\chaptertitlename}章}  
%\titleformat{\chapter}[hang]{\vspace{-2.5cm}\centering}{\sanhao 第 \chaptertitlename 章 \,}{1em}{\sanhao\hei}[\vspace{-10mm}]
\renewcommand\appendix{\setcounter{secnumdepth}{-2}}
\newcommand{\sectionname}{节}   
\renewcommand{\appendix}{附~录}  
\renewcommand{\listfigurename}{图~目~录}  
\renewcommand{\listtablename}{表~目~录}  
\renewcommand{\indexname}{索~引}  
%\newcommand{\keywords}[1]{\\ \\ \textbf{关~键~词}:#1}   段首空两格
\newcommand{\keywords}[1]{\textbf{关~键~词}:#1}  
%\titleformat{\chapter}[block]{\center\Large\bf}{\chaptertitlename}{20pt}{}  
%\renewcommand\abstractname{\Large \bfseries \CJKfamily{ZenHei}摘\ 要}		% 摘要 空两格,
\renewcommand\abstractname{\Large \CJKfamily{ZenHei}摘\ 要}   % 摘要 ,
\renewcommand{\figurename}{\CJKfamily{ZenHei} 图} 			% 图
\renewcommand{\tablename}{\CJKfamily{ZenHei} 表}            % 表
\renewcommand\refname{{\fontsize{10.5}{12.6} \selectfont \CJKfamily{KT} \textbf{【参考文献】}}} % 参考文献
\renewcommand\contentsname{\centerline{\CJKfamily{ZenHei} 目~录}}	%目录居中
\renewcommand{\today}{\number\year 年 \number\month 月 \number\day 日}	%中文日期
%\renewcommand{\theequation}{\arabic{chapter}-\arabic{equation}}

%========================================
%		Header Settings
\pagestyle{fancy}			%
\chead{\textcolor{gray}{\small{深圳大学本科毕业论文——}}}	% 页眉中部
\lhead{}		% 页眉左部,设为空
\rhead{}		% 页眉右部,设为空

%========================================
